\documentclass[uplatex, onecolumn, 10pt]{jsarticle}

\usepackage[dvipdfmx]{graphicx}
\usepackage{latexsym}
\usepackage{bmpsize}
\usepackage{url}
\usepackage{comment}

\def\Underline{\setbox0\hbox\bgroup\let\\\endUnderline}
\def\endUnderline{\vphantom{y}\egroup\smash{\underline{\box0}}\\}

\newcommand{\ttt}[1]{\texttt{#1}}

\begin{document}

\title{\vspace{-40mm}\bf{\LARGE{ゼミ報告書}}}
\author{\vspace{-40mm}有馬 温 60220109}
\date{2025-05-30 Fri}
\maketitle


\section{今日までにやったこと}

\subsection*{研究関連} 
\begin{itemize}
	\item 研究テーマ選定
		\begin{itemize}
			\item 書籍「テスト駆動開発」(第3部途中)
			\item 論文「GPTを用いたJava自動テストツールJARTGPTの試作」(根木原さん)
			\item 論文「ソフトウェアテストにおける静的コード解析ツールの段階的適用による\\
            不具合修正作業の効率化」奈良先端科学技術大学院大学 (読み途中)
        \end{itemize}
\end{itemize}

\subsection*{その他}
\begin{itemize}
	\item ロボコン MTG
	\item ロボコン タスク
	\begin{itemize}
        \item 「カラーセンサから値を取得する」
        \item 「直進の親クラスを作成する」
    \end{itemize}
	\item ハッカソン 環境構築
	\item codelessさんとの顔合わせ
	\item 宮崎大学大学院修士課程 進学説明会に参加
\end{itemize}


\section{次までにやること}

\subsection*{研究関連}
\begin{itemize}
	\item テストケース関連の研究
	\begin{itemize}
        \item テストについての勉強
        \item 先行研究や論文の調査
    \end{itemize} 
    \item ほかの研究テーマについても考えてみる
\end{itemize}

\subsection*{その他}
\begin{itemize}
    \item ロボコン MTG
    \item ロボコン タスク
	\begin{itemize}
        \item 「直進の子クラスを作成する」
    \end{itemize}
    \item ハッカソン (5/31 - 6/1)
    \item 第2回フリープレゼンの資料作成
    \item 5月のノートまとめ
\end{itemize}

\end{document}
