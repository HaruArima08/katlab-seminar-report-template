\documentclass[uplatex, onecolumn, 10pt]{jsarticle}

\usepackage[dvipdfmx]{graphicx}
\usepackage{latexsym}
\usepackage{bmpsize}
\usepackage{url}
\usepackage{comment}

\def\Underline{\setbox0\hbox\bgroup\let\\\endUnderline}
\def\endUnderline{\vphantom{y}\egroup\smash{\underline{\box0}}\\}

\newcommand{\ttt}[1]{\texttt{#1}}

\begin{document}

\title{\vspace{-40mm}\bf{\LARGE{ゼミ報告書}}}
\author{\vspace{-40mm}有馬 温 60220109}
\date{2025-06-06 Fri}
\maketitle


\section{今日までにやったこと}

\subsection*{研究関連} 
\begin{itemize}
	\item 研究テーマの選定
		\begin{itemize}
			\item 論文「ソフトウェアテストにおける静的コード解析ツールの段階的適用による\\
            不具合修正作業の効率化」奈良先端科学技術大学院大学 
        \end{itemize}
        \begin{itemize}
			\item テスト駆動開発
			\begin{itemize}
                \item ロボコンタスクなどでテスト駆動の考え方で実践してみた。実装後のレビューで足りないテストケースがあり、なかなか難しかったがテストを意識した実装は重要だと感じた。
            \end{itemize}
        \end{itemize}
        \begin{itemize}
			\item テスト系の勉強(先輩方の論文、テストツールまるわかりガイド(入門編))
        \end{itemize}
\end{itemize}

\subsection*{その他}
\begin{itemize}
	\item ロボコン MTG、タスク
	\item ハッカソン
\end{itemize}


\section{次までにやること}

\subsection*{研究関連}
\begin{itemize}
	\item 研究テーマの選定
	\begin{itemize}
        \item テスト(テストケース)についての勉強
        \begin{itemize}
            \item 論文読み
            \item 先行研究調査
        \end{itemize}
    \end{itemize}
\end{itemize}

\subsection*{その他}
\begin{itemize}
    \item ロボコン MTG、タスク
    \item バイト関連
    \item 第2回フリープレゼンの資料作成
\end{itemize}

\end{document}
