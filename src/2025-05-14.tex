\documentclass[uplatex, onecolumn, 10pt]{jsarticle}

\usepackage[dvipdfmx]{graphicx}
\usepackage{latexsym}
\usepackage{bmpsize}
\usepackage{url}
\usepackage{comment}

\def\Underline{\setbox0\hbox\bgroup\let\\\endUnderline}
\def\endUnderline{\vphantom{y}\egroup\smash{\underline{\box0}}\\}

\newcommand{\ttt}[1]{\texttt{#1}}

\begin{document}

\title{\vspace{-40mm}\bf{\LARGE{ゼミ報告書}}}
\author{\vspace{-40mm}有馬 温 60220109}
\date{2025-05-14 Wed}
\maketitle


\section{今日までにやったこと}

\subsection*{研究関連} 
\begin{itemize}
	\item 論文調査
		\begin{itemize}
			\item 「機械学習を用いた仕様書からのテストケース自動生成ツールSpec2Testの試作」平木場さん
			\item 「テストコード生成に大規模言語モデルを用いたJava 自動テストツール JARTGPT の試作」\\ 根木原さん
			\item 「大規模言語モデルの活用によるテスト駆動開発の継続的支援を目的としたフレームワークCATddの開発」 宮下さん
			\item 「生成 AI を活用したテスト設計に関する考察」jstage
		\end{itemize}
	\item CATddの環境構築
	\item 研究テーマ 情報収集
		\begin{itemize}
			\item テスト駆動開発, 自動テストケース生成
		\end{itemize}
\end{itemize}

\subsection*{その他}
\begin{itemize}
	\item フリープレゼン スライド準備
	\item ロボコン 勉強会, 環境構築
	\item React 勉強
\end{itemize}


\section{次までにやること}

\subsection*{研究関連}
\begin{itemize}
	\item 研究テーマ 情報収集
		\begin{itemize}
			\item テスト駆動開発, テストに関する研究について調べる
		\end{itemize}
	\item CATdd 
		\begin{itemize}
			\item 実際にCATddを動かしてみる
		\end{itemize}
\end{itemize}

\subsection*{その他}
\begin{itemize}
    \item ロボコン MTG , 資料読み
    \item c++ 勉強
    \item React 勉強
\end{itemize}

\end{document}
